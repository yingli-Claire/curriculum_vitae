\documentclass{SimpleResume}

\name{卡尔·海因里希·马克思}{Karl Heinrich Marx}
\bio{%
    \birthday  {1818年05月05日}%
    \university{柏林大学}%
    \degree    {哲学博士}%
    \email[2]  {karl\_marx\_did\_not\_have\_an\_email@example.com}% 2倍宽度
    \info      {\faMapMarked}{德国 / 普鲁士}% 默认标签以外的标签
    \address[2]{莱茵兰-普法尔茨州特里尔市布吕肯街10号}% 2倍宽度,不换行
    \address   {莱茵兰-普法尔茨州特里尔市布吕肯街10号}% 1倍宽度,可换行
}
\photo{Karl_Marx.png}{3cm}    % 照片宽度

\begin{document}
    \makeheader

    \moduleEdu{教育经历}
        \begin{timeline}
            \event{1841.04}{%
                耶拿大学哲学系%
            }
            \event{1836.10至1841.03}{%
                \textbf{\cell{柏林大学}\cell{德国柏林}\cell{哲学博士}}\par
                \textbf{\cell[1/4]{柏林大学}}\cell[1/4]{德国柏林}\cell[1/4]{法律系}\cell[1/4]{哲学博士}% 设置格子宽度
                % 默认宽度:1/3;可选宽度:1/4、1/2、2/3、3/4
            }
            \event{1835.10至1836.10}{%
                波恩大学法律系%
            }
            \event{1830.10至1835.09}{%
                特里尔中学%
            }
        \end{timeline}

    \moduleAca{论文发表}

    \begin{itemize}
        \item \textbf{马克思}:《1848年至1850年的法兰西阶级斗争》,《新莱茵报。政治经济评论》,1850年
        \item \textbf{马克思}:《关于费尔巴哈的提纲》,1845年
        \item \textbf{马克思}:《论德谟克利特的自然哲学和伊壁鸠鲁的自然哲学之间的差别》,博士毕业论文
        \item \textbf{马克思}:《青年在选择职业时的考虑》,中学毕业作文
    \end{itemize}

    \moduleBk{专著}
        \begin{itemize}
            \item \textbf{马克思}:《法兰西内战》,1871年
            \item \textbf{马克思}:《资本论》(第一卷),1867年
            \item \textbf{马克思}:《路易·波拿巴的雾月十八日》,1852年
            \item \textbf{马克思},恩格斯:《共产党宣言》,1848年
            \item \textbf{马克思},恩格斯:《德意志意识形态》,1846年
            \item \textbf{马克思}:《黑格尔法哲学批判》,1843年
        \end{itemize}

    \moduleCfr{参加会议}
        \begin{timeline}
            \event{1871.09.25}{%
                国际工人协会成立七周年庆祝会%
            }
            \event{1864.09.28}{%
                国际工人协会成立大会%
            }
            \event{1847.11.29至12.08}{%
                共产主义者同盟第二次代表大会\par
                \begin{details}
                    \item 马克思和恩格斯在大会上阐述了科学共产主义的观点。
                    \item 大会委托他们以宣言的形式拟定纲领。
                    \item 1848年2月21日,《共产党宣言》德文单行本在伦敦问世,24日正式出版。
                \end{details}%
            }
            \event{1846.03.30}{%
                布鲁塞尔共产主义通讯委员会会议%
            }
        \end{timeline}

    \moduleTit{职位和头衔}
        \begin{timeline}
            \event{1866.09}{%
                国际工人协会总委员会委员%
            }
            \event{1848.03}{%
                共产主义者同盟中央委员会主席%
            }
            \event{1847.11}{%
                布鲁塞尔民主协会副主席%
            }
        \end{timeline}

    \moduleHob{兴趣爱好}
        \begin{form}
            \event{学术兴趣}{%
                读书;%
                研究数学问题。%
            }
            \event{业余爱好}{%
                参加英国集市年度骑驴竞速比赛。%
            }
        \end{form}   

    % 默认主题以外的标题,需指定小图标
    \module{家庭状况}{\faHome}
        \begin{form}
            \event{父母}{%
                父亲:海因里希·马克思(Heinrich Marx);%
                母亲:亨丽埃塔·普莱斯堡(Henrietta Pressburg)。%
            }
            \event{兄弟姐妹}{%
                莫里茨(Moritz)、苏菲(Sophie)、赫尔曼(Hermann)、亨列特(Henriette)、路易斯(Louise)、艾米莉(Emilie)和卡罗琳(Karoline)等。%
            }
            \event{婚姻}{%
                妻子:燕妮·马克思(Jenny Marx)。%
            }
        \end{form}

    \moduleSta{个人陈述}
        \begin{parsCJK}
            一个选择了自己所珍视的职业的人,一想到他可能不称职时就会战战兢兢——这种人单是因为他在社会上所居地位是高尚的,他也就会使自己的行为保持高尚。

            在选择职业时,我们应该遵循的主要指针是人类的幸福和我们自身的完美。不应认为,这两种利益是敌对的,互相冲突的,一种利益必须消灭另一种的;人类的天性本来就是这样的:人们只有为同时代人的完美、为他们的幸福而工作,才能使自己也达到完美。

            如果一个人只为自己劳动,他也许能够成为著名的学者、大哲人、卓越诗人,然而他永远不能成为完美无疵的伟大人物。

            历史承认那些为共同目标劳动因而自己变得高尚的人是伟大人物;经验赞美那些为大多数人带来幸福的人是最幸福的人;宗教本身也教诲我们,人人敬仰的理想人物,就曾为人类牺牲了自己——有谁敢否定这类教诲呢?

            如果我们选择了最能为人类福利而劳动的职业,那么,重担就不能把我们压倒,因为这是为大家而献身;那时我们所感到的就不是可怜的、有限的、自私的乐趣,我们的幸福将属于千百万人,我们的事业将默默地、但是永恒发挥作用地存在下去,面对我们的骨灰,高尚的人们将洒下热泪。
        \end{parsCJK}

\end{document}
