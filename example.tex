% !TeX encoding = UTF-8
% !TeX program = xelatex
% !TeX spellcheck = en_US
% choose Xelatex compiler 选择Xelatex进行编译
\documentclass{MyResume}

\usepackage[English]{languageSelection}

\begin{document}
\pagenumbering{gobble}

% 填写你的名字
\CN{
  \name{方鸿渐}
}
\EN{
  \name{Jianhong Fang}
}
%**********************************相关信息****************************************
% \otherInfo后面的四个大括号里的所有信息都会在一行输出,最多使用四个大括号,填写四个信息
% 如果选择不填信息,那么大括号必须空着不写,而不能删除大括号。
% 如果想要把信息写两行,那就用两次指令\otherInfo{}{}{}{}即可
\CN{
  \info{手机:(+86) 1234567890}{邮箱:test@test.test}{}{}
  \info{性别:男}{籍贯:江南}{}{}
  \info{来历:钱钟书《围城》}{}{}{}
}
\EN{
  \info{mobile: (+86) 1234567890}{email: test@test.test}{}{}
  \info{Gender: Male}{Hometown: South China}{}{}
  \info{Origin: Fortress Besieged}{}{}{}
}


\CN{
  \section{教育背景}
}
\EN{
  \section{EDUCATION}
}
%***********列举*********************
%***可添加多个\item,得到多个列举项,类似的也可以用\textbf{}、\textit{}做强调
\CN{
  \begin{itemize} [parsep=1ex]
    \item \textbf{证书来源}:购买自爱尔兰商人
  \end{itemize}
}
\CN{
  \datedsubsection{\textbf{北平某大学},实验班,\textit{本科}}{1922.09 - 1926.06}
}
\CN{
  \begin{itemize} [parsep=1ex]
    \item \textbf{土木工程系}:不喜欢,转系
    \item \textbf{社会学系}:不喜欢,转系
    \item \textbf{中国文学系}:从此毕业
  \end{itemize}
}
\EN{
  \begin{itemize} [parsep=1ex]
    \item \textbf{Certificate source}: purchased from an Irish businessman
  \end{itemize}
}
\EN{
  \datedsubsection{\textbf{A university in Beiping}, experimental class, \textit{Undergraduate}}{1922.09 - 1926.06}
  \begin{itemize} [parsep=1ex]
    \item \textbf{Department of Civil Engineering}: I don't like it, change department
    \item \textbf{Department of Sociology}: I don't like it, change department
    \item \textbf{Department of Chinese Literature}: graduated
  \end{itemize}
}

\CN{
\section{简历写作注意事项}

写作时不要泛泛而谈太笼统,要应用STAR原则,即Situation(情景)、Task(任务)、Action(行动)和Result(结果)四个英文单词的首字母组合。

\begin{itemize}[parsep=0.5ex]
  \item S指的是situation,事情是在什么情况下发生
  \item T指的是task,你是如何明确你的目标的
  \item A指的是action,针对这样的情况分析,你采用了什么行动方式
  \item R指的是result,结果怎样,在这样的情况下你学习到了什么
\end{itemize}
}

\EN{
\section{Resume writing notes}


\begin{itemize}[parsep=0.5ex]
  \item S refers to the situation, under what circumstances did things happen
  \item T refers to the task, how do you define your goal
  \item A refers to the action, what kind of action did you take in this situation analysis
  \item R refers to result, what is the result, what did you learn in this situation
\end{itemize}
}


\end{document}
